
\section{System Analysis} \label{sec:analysis}

This section deals with the analysis of the aforementioned system of a wind turbine.
First, the description of the ODE system, which is insufficient for an implementation, is presented (\autoref{sec:analysis:ODE}) and the problems encountered are briefly described.

Second, the implementation of the system via transfer function is introduced \autoref{sec:analysis:tf}.
All important steps to implement the system in Matlab are shown.
By using figures published by Fragoso et al.– \cite{Fragoso_et_al_2017} the implementation in Matlab is verified.

The following chapter (\autoref{sec:condes}) is dealing with controller design. One important step of controller design is the relative gain array, which is conducted at the end of tis section.

\subsection{Simulation of ODE system} \label{sec:analysis:ODE}

Fragoso et al.~ \cite{Fragoso_et_al_2017} present a system of ODE for the wind turbine model.
As described in \autoref{sec:intro:model_eq} the model is subdivided into two parts.
The electrical subsystem is based in on two algebraic equations (\autoref{eq:intro:tau_em}, \autoref{eq:intro:P_g}).

For the mechanical part an ODE is given
\begin{align}
    J_t \frac{d \omega_t}{dt} = \tau_a - \tau_{em}, \label{eq:analysis:ODE_mechanical}
\end{align}
where $\tau_a$ is the aerodynamic torque and defined as
\begin{align}
    \tau_a = \frac{1}{2} \rho \pi R^3 \frac{C_p\left( \lambda, \beta\right)}{\lambda} v^2.
\end{align}
The power coefficient $C_p$ is only described by a figure (see Figure 4 in \cite{Fragoso_et_al_2017}).
In Fragosos work \cite[p. 54 ff.]{Fragoso_PhD_2016} a function for $C_p$ is derived and also a figure shows the global course of the function (see Figure 3.8, p. 56).
In this function nine heuristic parameters are used to determine the characteristics.
It is unclear if this function is also used in the work of Fragoso et al.~ or not.

In the appendix of Fragosos PhD-thesis there are tabulated value for $C_p$.
These were implemented as a lookuptable in Matlab.
By using the lookup table, values for $C_p$ could be determined, but the solutions of the ODE were not in a realistic range of values.

\begin{lstlisting}[style=Matlab-editor,caption={Matlab function for lookuptable. With this function we can get a corresponding $C_p$ value for given $\beta$ and $\lambda$. The \texttt{.csv}-file can be found in the GitHub repository under},captionpos=b,label={list:analysis:lookup}]
% Loading tabulated values as .csv file
p.lookup = readtable('lookuptable_c_p.csv');
% Setting corresponding column names 
p.lookup.Properties.VariableNames = {'wind_speed','rotor_speed','U','i_a','P_e','P_m','lambda','C_p','C_q','beta'};

% Looking up values for values of lambda and beta
function C_p = power_coefficient(lambda,beta,lookup)
    C_p = interp1(lookup.lambda(lookup.beta==beta),lookup.C_p(lookup.beta==beta),lambda);
end
\end{lstlisting}

Even if it is possible to get correct values for $C_p$, we still have only one ODE and a algebraic equation.
This is not representing a systems with two systems states.

Due to the problems mentioned here, the representation and solution of the system in state space was not pursued further and the transfer function mentioned in the Fragose et al.~ was used instead.

\subsection{Implementation of transfer function model} \label{sec:analysis:tf}

All controllers will be designed based on a linear model.
For different wind speeds $v$ Fragoso et al.~ identified five different linear models.
The pitch angel $\beta$ and the duty cycle $\alpha$ are the systems states.
The outputs are the rotational speed of the blade $\omega_r$ and the generated electric power $P_g$.
\begin{align}
    \begin{pmatrix}
        \omega_r \\
        P_g
      \end{pmatrix} = 
      G(s) 
      \begin{pmatrix}
        \beta \\
        \alpha
      \end{pmatrix} + G_D(s) v
\end{align}

The transferfunctions $G$ and $G_D$ where identified for windspeed $v=\left[6,7,8,9,10 \right] \si{\metre\per\second}$.
The results can be found in \cite[Table 2]{Fragoso_et_al_2017} and will not be reprinted here.

\begin{table}[H]
    \label{tab:analysis:import_var}
    \caption{Compilation of the most important variables, their meaning, unit and value range. The value ranges in italics are estimated values derived from figures. In addition, only the deviation $\Delta v$ is given, as we consider the disturbance in the following only as a deviation from zero. That is why we also consider a maximum value range between -1 and 1, as we would enter the \textit{domain} of another transfer function for larger values.}
    \centering
    \begin{tabular}{cllcc} \toprule
        Variable & Name & Interpration & Unit & Range \\ \midrule
        $\Delta v$ & change in windspeed & disturbance  & \si{\metre\per\second} & $\mathit{\left[-1,1\right]}$\\
        $\beta$ & pitch angle & state & \si{\degree}& $\left[0,25\right]$ \\
        $\alpha$ & duty cycle& state & 1 & $\left[0,1\right]$\\
        $\omega_r$ & rotational speed of blade & input & rpm & $\mathit{\left[1400,2000 \right]}$  \\
        $P_g$ &generated electric power &input & \si{\watt} &  $\mathit{\left[3,9 \right]}$\\ \bottomrule
    \end{tabular}
\end{table}

\subsubsection*{Implementation in Matlab}

To generate the transfer functions in Matlab the \texttt{tf}-command and the \texttt{table} function are used.

\begin{lstlisting}[style=Matlab-editor,caption={},captionpos=b,label={list:analysis:tf}]

% Wind speed: 6 m/s
G_S_num_6 = {-8.059,-405.4;-0.0081479,4.4195};
G_S_denom_6 = {[75.27,17.35,1],[28.25, 16.47, 1];[0, 16.873, 1],[0, 1.7589, 1]};
G_D_num_6 = {364.9; 1.139};
G_D_denom_6 = {[118.3, 21.76, 1];[65.28, 17.09, 1]};

... Wind speeds 7,8,9 an 10 m/s ...

% Making a table to store all transfer_function
varNames = ["wind_speed","G_S_numerator","G_S_denominator","G_D_numerator","G_D_denominator","G_S","G_D","RGA"];
table_tf = table;

% Filling the table with numerator and denominator
table_tf(1,1:5) = {6,G_S_num_6,G_S_denom_6,G_D_num_6,G_D_denom_6};
...
table_tf(5,1:5) = {10,G_S_num_10,G_S_denom_10,G_D_num_10,G_D_denom_10};
table_tf.Properties.VariableNames = varNames(1:5);

% For loop for creating transfer function by using tf-command
for i = 1:5
    table_tf{i,6} = {tf(table_tf{i,"G_S_numerator"}{:},table_tf{i,"G_S_denominator"}{:})};
    table_tf{i,7} = {tf(table_tf{i,"G_D_numerator"}{:},table_tf{i,"G_D_denominator"}{:})};
    table_tf{i,8} = {NaN};
end

% Naming the columns in the table
table_tf.Properties.VariableNames = varNames;

\end{lstlisting}

\subsubsection{Verifying the implementation}

\subsection{Relative Gain Array}  \label{sec:analysis:RGA}

\subsubsection*{Implementation in Matlab}

\subsubsection*{Results of RGA}