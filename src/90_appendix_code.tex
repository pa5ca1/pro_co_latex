\section{Matlab Code}

\subsection{Ziegler-Nichols-Methods} \label{app:ZN}

\subsubsection{Computing infliction point} \label{app:ZN:infl_point}

\begin{lstlisting}[style=Matlab-editor,caption={Matlab code for calculating the infliction point.},captionpos=b,label={list:app:infliction_point}]
function [slope, intcpt] = calc_infliction_point_slope(y,t)
    %% Find the inflection point and tangent
    % Source: https://de.mathworks.com/matlabcentral/answers/295156-how-to-find-the-inflection-point-of-a-curve-in-matlab
    % gradient of step response
    dydt = gradient(y);
    % Find y and t of highest gradient
    % Get only the unique values
    [dydt,i_unique,~] = unique(dydt,'stable');
    t_out = t(i_unique);
    y_out = y(i_unique);
    t_infl = interp1(dydt, t_out, max(dydt));
    y_infl = interp1(t_out, y_out, t_infl);
    % Calculate slope
    slope  = interp1(t_out, dydt, t_infl);
    % Find y-interception
    intcpt = y_infl - slope*t_infl;

end
\end{lstlisting}

\subsubsection{Computing control parameters} \label{app:ZN:control_para}

\begin{lstlisting}[style=Matlab-editor,caption={Computation of $T_d$, $T$ and $K$ in Matlab.},captionpos=b,label={list:app:controller_parameters}]
function [K, T, T_d,t_interception] = calc_gain_time_const_time_delay(y,t,tngt)

    % %Find the T_d and T
    % Calculate the intercept of slope with K
    K = max(y);
    % Get t-value when tangents hits K
    % First we calculate the index in t. With this index the time t
    % can be calculated
    t_index_interception = find(tngt>=max(y),1,'first');
    t_interception = t(t_index_interception);
    t_index_T_d = find(tngt>=0,1,'first');
    % Calculation of T_d and T
    T_d = t(t_index_T_d);
    T = t_interception-T_d;
    
    
end
\end{lstlisting}


\subsubsection{Two-point method} \label{app:ZN:2p_method}

\begin{lstlisting}[style=Matlab-editor,caption={Implementation of two-point method in Matlab.},captionpos=b,label={list:app:two_point_method}]
[y,t] = step(G,t_linspace);
% Get maximum value of y
y_inf = max(y);
y_0 = min(y);
y_hat =  y_inf-y_0;
y_28 = y_0 + 0.28 * y_hat;
y_63 = y_0 + 0.63 * y_hat;
% Making y unique valued
[y_out,i_unique,~] = unique(y,'stable');
t_out = t(i_unique);
% Calculation of time points where y reaches 28/63 percent
t_28 = interp1(y_out,t_out,y_28);
t_63 = interp1(y_out,t_out,y_63);

%% Calculation of T, T_d
T = 3/2 * (t_63-t_28);
T_d = t_63 - T;
K = y_inf;

\end{lstlisting}